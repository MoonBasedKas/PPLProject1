
\documentclass[sigconf, nonacm, authorversion, language=english, 12pt]{acmart}
\usepackage{setspace} \doublespacing 

\def\tab{\hspace*{5mm}}
\AtBeginDocument{%
  \providecommand\BibTeX{{%
    Bib\TeX}}}

%% Rights management information.  This information is sent to you
%% when you complete the rights form.  These commands have SAMPLE
%% values in them; it is your responsibility as an author to replace
%% the commands and values with those provided to you when you
%% complete the rights form.
\setcopyright{acmlicensed}
\copyrightyear{2018}
\acmYear{2018}
\acmDOI{XXXXXXX.XXXXXXX}


\acmConference[Conference acronym 'XX]{Make sure to enter the correct
  conference title from your rights confirmation emai}{June 03--05,
  2018}{Woodstock, NY}


\begin{document}


\title{THE TITLE}


\author{Kassidy Maberry}
\email{kassidy.maberry@student.nmt.edu}
\affiliation{%
  \institution{New Mexico Institue of Mining and Technology}
  \streetaddress{P.O. Box 3224}
  \city{Socorro}
  \state{New Mexico}
  \country{USA}
  \postcode{87081}
}



\newcommand{\PLtitle}{\setlength{\parindent}{0pt}
  \twocolumn[{%
        \begin{center}
          \huge\textbf{THE TITLE}\\
          \normalsize Subtitle\\

          \begin{singlespace}


            Name: Kassidy Maberry\\
            Department of Computer Science and Engineering\\
            New Mexico Institue of Mining and Technology\\
            Socorro, New Mexico, USA\\
            kassidy.maberry@student.nmt.edu\\
            \bigskip
            \bigskip
          \end{singlespace}
        \end{center}
      }]
}

%%
%% The abstract is a short summary of the work to be presented in the
%% article.
\begin{abstract}
  Insert abstract
\end{abstract}


%%
%% The code below is generated by the tool at http://dl.acm.org/ccs.cfm.
%% Please copy and paste the code instead of the example below.
%%
\begin{CCSXML}
  <ccs2012>
  <concept>
  <concept_id>10011007.10011006.10011008.10011024</concept_id>
  <concept_desc>Software and its engineering~Language features</concept_desc>
  <concept_significance>500</concept_significance>
  </concept>
  </ccs2012>
  <concept>
  <concept_id>10011007.10011006.10011039.10011311</concept_id>
  <concept_desc>Software and its engineering~Semantics</concept_desc>
  <concept_significance>500</concept_significance>
  </concept>
  <concept>
  <concept_id>10011007.10011006.10011039.10011040</concept_id>
  <concept_desc>Software and its engineering~Syntax</concept_desc>
  <concept_significance>300</concept_significance>
  </concept>
\end{CCSXML}

\ccsdesc[300]{Software and its engineering~Semantics}
\ccsdesc[100]{Software and its engineering~Syntax}
\ccsdesc[500]{Software and its engineering~Language features}

%%
%% Keywords. The author(s) should pick words that accurately describe
%% the work being presented. Separate the keywords with commas.
\keywords{Do, Not, Us, This, Code, Put, the, Correct, Terms, for,
  Your, Paper}



%%
%% This command processes the author and affiliation and title
%% information and builds the first part of the formatted document.

% \maketitle
\PLtitle


\section*{Abstract}
Python, C++, and Java are all popular languages. [Finish it].\\

\section*{CCS Concepts}
\textbullet\hspace*{0.5mm} \textbf{Software and its engineering} $\rightarrow$ \textbf{Language Features}; Syntax; \textit{Semantics}.\\
\section*{Keywords}
Syntax, Semantics, Language Features.\\
\section{Introduction}
Python, C++, and Java all have their own unique advantage. That might make one prefer over the others.
In this we will attempt to determine the best features that each language has and what features we should remove in future iterations.

\section{Classes}
All the languages we are analyzing provide classes and are an important feature when designing a high level language. However, Python implements classes with unique
syntax and Semantics. Attributes not defined in classes are static private variables are declared with a double underscore $(\_\_x)$, and local
attributes must be declared in methods. Finally, constants do not exist. Being able to declare new attributes in a method is a very powerful feature.
The problem is that while powerful it hurts readability and is a weird choice for python as a language. When attributes could be defined anywhere it makes attributes hard to find.\\

In C++ and Java the classes are implemented similarly when considering attributes. All attributes are declared outside of methods. You state if the attribute is public, private, or static.
In python attributes and methods are public by default meanwhile Java and C++ attributes are Private by default. Private by default attributes is great. Enforcing abstraction on the user and
increasing the security of their classes as the user is forced to decide what can be accessed from the outside. Python declaring private is more readable but is held back by most attributes
most likely being declared within methods.\\



\section{Functions}

\section{Scoping}

\section{Control Flow}

\section{Data Types}

%  Could combine that with Classes and functions.

\section{Efficiency}
% Compi9lation and Platform dependency.

\end{document} % it will say there is a problem but there is not.

%%
%% End of file `sample-sigconf-i13n.tex'.
