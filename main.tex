
\documentclass[sigconf, nonacm, authorversion, language=english, 12pt]{acmart}
\usepackage{setspace} \doublespacing 

\def\tab{\hspace*{5mm}}
\AtBeginDocument{%
  \providecommand\BibTeX{{%
    Bib\TeX}}}

%% Rights management information.  This information is sent to you
%% when you complete the rights form.  These commands have SAMPLE
%% values in them; it is your responsibility as an author to replace
%% the commands and values with those provided to you when you
%% complete the rights form.
\setcopyright{acmlicensed}
\copyrightyear{2018}
\acmYear{2018}
\acmDOI{XXXXXXX.XXXXXXX}
\usepackage{enumitem}
\usepackage{listings}
\usepackage{xcolor}

\definecolor{codegreen}{rgb}{0,0.6,0}
\definecolor{codegray}{rgb}{0.5,0.5,0.5}
\definecolor{codepurple}{rgb}{0.58,0,0.82}
\definecolor{backcolour}{rgb}{0.95,0.95,0.92}

\lstdefinestyle{mystyle}{
    backgroundcolor=\color{backcolour},   
    commentstyle=\color{codegreen},
    keywordstyle=\color{magenta},
    numberstyle=\tiny\color{codegray},
    stringstyle=\color{codepurple},
    basicstyle=\ttfamily\footnotesize,
    breakatwhitespace=false,         
    breaklines=true,                 
    captionpos=b,                    
    keepspaces=true,                 
    numbers=left,                    
    numbersep=5pt,                  
    showspaces=false,                
    showstringspaces=false,
    showtabs=false,                  
    tabsize=2
}

\acmConference[Conference acronym 'XX]{Make sure to enter the correct
  conference title from your rights confirmation emai}{June 03--05,
  2018}{Woodstock, NY}


\begin{document}


\title{A Comparison of Languages}


\author{Kassidy Maberry}
\email{kassidy.maberry@student.nmt.edu}
\affiliation{%
  \institution{New Mexico Institue of Mining and Technology}
  \streetaddress{P.O. Box 3224}
  \city{Socorro}
  \state{New Mexico}
  \country{USA}
  \postcode{87081}
}



\newcommand{\PLtitle}{\setlength{\parindent}{0pt}
  \twocolumn[{%
        \begin{center}
          \huge\textbf{A Comparison of Languages}\\
          \normalsize A Comparision between C++, Python, and Java.\\

          \begin{singlespace}


            Name: Kassidy Maberry\\
            Department of Computer Science and Engineering\\
            New Mexico Institue of Mining and Technology\\
            Socorro, New Mexico, USA\\
            kassidy.maberry@student.nmt.edu\\
            \bigskip
            \bigskip
          \end{singlespace}
        \end{center}
      }]
}

%%
%% The abstract is a short summary of the work to be presented in the
%% article.
\begin{abstract}
  Insert abstract
\end{abstract}


%%
%% The code below is generated by the tool at http://dl.acm.org/ccs.cfm.
%% Please copy and paste the code instead of the example below.
%%
\begin{CCSXML}
  <ccs2012>
  <concept>
  <concept_id>10011007.10011006.10011008.10011024</concept_id>
  <concept_desc>Software and its engineering~Language features</concept_desc>
  <concept_significance>500</concept_significance>
  </concept>
  </ccs2012>
  <concept>
  <concept_id>10011007.10011006.10011039.10011311</concept_id>
  <concept_desc>Software and its engineering~Semantics</concept_desc>
  <concept_significance>500</concept_significance>
  </concept>
  <concept>
  <concept_id>10011007.10011006.10011039.10011040</concept_id>
  <concept_desc>Software and its engineering~Syntax</concept_desc>
  <concept_significance>300</concept_significance>
  </concept>
\end{CCSXML}

\ccsdesc[300]{Software and its engineering~Semantics}
\ccsdesc[100]{Software and its engineering~Syntax}
\ccsdesc[500]{Software and its engineering~Language features}

%%
%% Keywords. The author(s) should pick words that accurately describe
%% the work being presented. Separate the keywords with commas.
\keywords{Do, Not, Us, This, Code, Put, the, Correct, Terms, for,
  Your, Paper}



%%
%% This command processes the author and affiliation and title
%% information and builds the first part of the formatted document.

% \maketitle
\PLtitle


\section*{Abstract}
Python, C++, and Java are all popular languages each with their own advantages. We will observe how each
language compares with each other. Determining what we gain from each implementation and if a certain implementation is
better. With this analysis we can determine what features we would want to implement in future languages.\\

\section*{CCS Concepts}
\textbullet\hspace*{0.5mm} \textbf{Software and its engineering} $\rightarrow$ \textbf{Language Features}; Syntax; \textit{Semantics}.\\
\section*{Keywords}
Syntax, Semantics, Language Features.\\
\section{Introduction}
Python, C++, and Java all are popular to use languages.
Each language is unique and brings features and implementations that are unique to the language. These implementations might achieve similar goals,
and we will observe which implementation worked best for each goal and if the implementation was worth it.


\section{Class Attributes}

\tab All the languages we are analyzing provide classes and are an important feature when designing a high-level language. However, Python implements classes with unique
syntax and Semantics. Attributes not defined in classes are static private variables that are declared with a double underscore $(\_\_x)$, and local
attributes must be declared in methods [1]. Finally, constants do not exist. Being able to declare new attributes in a method is a very powerful feature.
The problem is that while powerful it hurts readability and is a weird choice for Python as a language. When attributes could be defined anywhere it makes attributes hard to find.\\

\tab In C++ and Java, the classes are implemented similarly when considering attributes. All attributes are declared outside of methods. You state if the attribute is public, private, or static.
In Python attributes and methods are public by default meanwhile Java and C++ attributes are Private by default [2][3]. Private by default attributes are great. Enforcing abstraction on the user and
increasing the security of their classes as the user is forced to decide what can be accessed from the outside. Python declaring private is more readable but is held back by most attributes
most likely being declared within methods.\\

\section{Class Inheritance}

\tab Another important aspect of classes is inheritance. Inheritance is very powerful and allows the user to reduce the amount of code they need to rewrite. A powerful feature extending from
inheritance is multiple inheritance. Java does not support multiple inheritance and forces the user to pick a singular class as the parent[2]. Python and C++ both support multiple inheritance
however each method is different. In multiple inheritances, you need to determine a way to deal with conflicting values and which inherited class takes priority. Python determines priority based
on which class is inherited first and any values that are shared in both classes will always be assigned as the first given class to inherent from [1]. C++ solves the issue by disallowing the user to have conflicting
method and attribute names. One way Java helps with the lack of multiple inheritance is by using interfaces. These serve as blueprints for classes and a class can implement multiple interfaces.
I believe C++ and Java solve the issue of multiple inheritance better than Python by being more secure; requiring the user to not have any ambiguity. C++ has the best method as it allows for the power of
multiple inheritances while keeping security.\\

\section{user-defined Data Types}
\tab Java requires the user to use classes [2] however Python and C++ do not require the user to. Not requiring classes is the better choice. Not every problem needs classes to be solved and requiring
classes can add unneeded complexity. While it's important for large-scale solutions it adds time and forces the user to think with objects in mind. C++ allows for two user-defined data types: the struct
and the class. Python will require classes for any user-defined data types. C++ provides the best solution by allowing the user to decide. You might not need the overhead of a class and instead, a struct
would be the better choice. Structs can also be paired with classes improving readability. \\

\tab Finally, all three languages allow the user to have enumeration. These allow for abstraction of something like the days
of the week without needing to overwork built-in data type. C++ forces the user to enumerate values to be integers [3]. Python allows enumerated values to be equal to different data types. For example,
Python allows for enumerations to be strings or integers. Finally, Java allows users to treat enumerations as if they were strings. I believe Python and Java allowing the user to treat enumerations as strings
is the better choice. It simplifies printing out an enumeration as you don't need to translate the enumeration using switch statements.\\

\section{Built-in Data Types}
\tab All languages support some of the same types. Python, Java, and C++ all support integers strings, boolean, and floats. C++ and Java share more
data types that are extensions of integers and floats. For example, We also have doubles, longs, unsigned integers, characters (Python treats characters as strings [1]), and shorts. While the generalization
of different types into a singular type is great for ease of use and security it hurts the efficiency of a language. For example, when using a singular character you don't need a lot of the power that comes with strings and
instead receive additional overhead. Using shorts instead of the default integers to save memory. Giving user choice is the best option.\\

\tab Java and C++ both support arrays. Arrays are of fixed size and type. Python instead opts to use Lists that are dynamically sized, and any type can be held within the same list. This is powerful but insecure
as you can create arrays where there are multiple types held within and a string and an integer do not have the same operations. Java implements this with the use of ArrayLists where you can hold any type
by creating an array list of type Object [2]. C++ can achieve this using a combination of void pointers and user-defined Data types since void pointers can point to anything [3]. All languages support strings but
C++, Java, and Python both have very powerful syntax for strings. Looking at the code below.\\
\begin{lstlisting}[language=Python]
  h = "hello"
  w = "world"
  str = h + w  
\end{lstlisting}
Being able to use the addition operation for string concatenation is great and improves readability. Python introduces this operation for Lists and Tuples as well [1].

\section{Control Flow}
\tab The control flow of a program is very important for reading the language. C++, Java, and Python all implement some control methods very similarly. The while loop is functionally the same however Java and C++
have the added do-while loop allowing the user to execute the code at minimum once. The for loop is in all three languages however Python's for loop is different from the rest. Python's will iterate through a list [1] (or any iterable)
where Java and C++ for loops are more advanced while loops where now you can define a variable to iterate and how to iterate after each pass. Python has match statements which are the same as switch statements in Java
C++. All languages contain the if statement however once again they are all similar. Python introduces the elif statement which serves as the if else. The elif statement is great as it shortens the syntax of if else without
changing the meaning. Finally, all three languages contain the try statement for exception handling.\\

\tab These are done through the Try-Catch statement. They are all implemented very similarly allowing the user to try a statement of code, catch and handle
exceptions, and throw user-created exceptions. This is a great feature for the robustness of the language. Finally, C++ has the goto statement.
This is a bad inclusion. The goto statement disrupts the flow of the program
and is insecure. The user can skip major important parts of a program, disrupting the program's control flow. This directly hurts the security and readability of code. The power
we gain is not worth the costs.\\



\section{Garbage Collection}

\tab While a lot of factors can be broken down into sections there are some factors existing outside these sections. These
factors are still worth considering. One instance is garbage collection if we want implicit or explicit. While implicit
garbage collection is great and helps secure the language. This could save users a lot of time and make debugging easier
but if you are looking for efficiency removing the overhead of garbage collection explicit would be better.\\

\tab Python and Java both have implicit garbage collection [1][2] meanwhile C++ has explicit [3]. Explicit garbage collection relies on
the user to properly manage their memory usage. Users are very likely to not properly manage their memory and this will
force the user to spend more time debugging. While the efficiency is great it is not worth the security we lose.\\

\section{Platform Dependency}
\tab C++ is not platform dependent while Python and Java are both platform independent. C++ code will not run for machines it has not been compiled for. Not all operating systems are the same
and thus a language designed for operating systems will have inconsistency. Python will always run consistently due to being an interpreted language [1]. As long as you have an interpreter
the language will always run the same. Java uses a combination of compilation and interpreting. Once again compiling to what Java called bytecode and then interpreting said code [2]. Finally, C++
will only compile [3]. Python tends to be slower due to being purely interpreted. C++ and Java depend on which will run faster. Java is capable of making runtime-specific optimizations.
This helps with Java being capable of running faster in some circumstances [4].\\



\section{Conclusion}
\tab After going over these various features if we were to design our language we should have some important features. For example, implicit garbage collection is a must-have, but we should avoid
the goto statement. The try-catch statement was an overall good addition all languages had that we should include. Classes should be included but optional and features like structs and enumeration should be
included for user choice. Class attributes should not be declared within the class and multiple inheritance should be included; to avoid ambiguity we should require the user to have
all methods and attributes have different names. We should include the option of a built-in dynamically sized data type like ArrayLists where all members must be the same type for security. All languages have their
strengths and if we were to make a new language we should combine some of the previously mentioned features.

\section*{References}
\begin{enumerate}[label={[\arabic*]}]
  \item Python Software Foundation. 2024. Python 3.12.2 Documentation. Retrieved March 17, 2024 from \href{https://docs.Python.org/3/}{https://docs.Python.org/3/}\\

  \item Oracle Corporation. 2024. Java Documentation. Retrieved March 17, 2024 from\\ \href{https://docs.oracle.com/en/java/}{https://docs.oracle.com/en/java/}\\

  \item Microsoft. 2024. C++ Language Documentation. Retrieved March 17, 2024 from\\ \href{https://learn.microsoft.com/en-us/cpp/cpp/?view=msvc-170}{https://learn.microsoft.com/en-us/cpp/cpp}\\

  \item David Lion, Adrian Chiu, Michael Stumm, and Ding Yuan. 2022. Investigating Managed Language Runtime Performance: Why JavaScript and Python are 8x and 29x slower than C++, yet Java and Go can be Faster?.
        2022 USENIX Annual Technical Conference (USENIX ATC 22). ISBN: 978-1-939133-29-40\\
\end{enumerate}


\end{document} % it will say there is a problem but there is not.

%%
%% End of file `sample-sigconf-i13n.tex'.
