
\documentclass[sigconf, nonacm, authorversion, language=english, 12pt]{acmart}
\usepackage{setspace} \doublespacing 


\AtBeginDocument{%
  \providecommand\BibTeX{{%
    Bib\TeX}}}

%% Rights management information.  This information is sent to you
%% when you complete the rights form.  These commands have SAMPLE
%% values in them; it is your responsibility as an author to replace
%% the commands and values with those provided to you when you
%% complete the rights form.
\setcopyright{acmlicensed}
\copyrightyear{2018}
\acmYear{2018}
\acmDOI{XXXXXXX.XXXXXXX}


\acmConference[Conference acronym 'XX]{Make sure to enter the correct
  conference title from your rights confirmation emai}{June 03--05,
  2018}{Woodstock, NY}


\begin{document}


\title{THE TITLE}


\author{Kassidy Maberry}
\email{kassidy.maberry@student.nmt.edu}
\affiliation{%
  \institution{New Mexico Institue of Mining and Technology}
  \streetaddress{P.O. Box 3224}
  \city{Socorro}
  \state{New Mexico}
  \country{USA}
  \postcode{87081}
}



\newcommand{\PLtitle}{\setlength{\parindent}{0pt}
  \twocolumn[{%
        \begin{center}
          \huge\textbf{THE TITLE}\\
          \normalsize Subtitle\\

          \begin{singlespace}


            Name: Kassidy Maberry\\
            Department of Computer Science and Engineering\\
            New Mexico Institue of Mining and Technology\\
            Socorro, New Mexico, USA\\
            kassidy.maberry@student.nmt.edu\\
            \bigskip
            \bigskip
          \end{singlespace}
        \end{center}
      }]
}

%%
%% The abstract is a short summary of the work to be presented in the
%% article.
\begin{abstract}
  Insert abstract
\end{abstract}


%%
%% The code below is generated by the tool at http://dl.acm.org/ccs.cfm.
%% Please copy and paste the code instead of the example below.
%%
\begin{CCSXML}
  <ccs2012>
  <concept>
  <concept_id>10011007.10011006.10011008.10011024</concept_id>
  <concept_desc>Software and its engineering~Language features</concept_desc>
  <concept_significance>500</concept_significance>
  </concept>
  </ccs2012>
  <concept>
  <concept_id>10011007.10011006.10011039.10011311</concept_id>
  <concept_desc>Software and its engineering~Semantics</concept_desc>
  <concept_significance>500</concept_significance>
  </concept>
  <concept>
  <concept_id>10011007.10011006.10011039.10011040</concept_id>
  <concept_desc>Software and its engineering~Syntax</concept_desc>
  <concept_significance>300</concept_significance>
  </concept>
\end{CCSXML}

\ccsdesc[300]{Software and its engineering~Semantics}
\ccsdesc[100]{Software and its engineering~Syntax}
\ccsdesc[500]{Software and its engineering~Language features}

%%
%% Keywords. The author(s) should pick words that accurately describe
%% the work being presented. Separate the keywords with commas.
\keywords{Do, Not, Us, This, Code, Put, the, Correct, Terms, for,
  Your, Paper}



%%
%% This command processes the author and affiliation and title
%% information and builds the first part of the formatted document.

% \maketitle
\PLtitle


\section*{Abstract}

\section*{CCS Concepts}
\textbullet\hspace*{0.5mm} \textbf{Software and its engineering} $\rightarrow$ \textbf{Language Features}; Syntax; \textit{Semantics}.\\
\section*{Keywords}

\section{Introduction}

\end{document} % it will say there is a problem but there is not.

%%
%% End of file `sample-sigconf-i13n.tex'.
